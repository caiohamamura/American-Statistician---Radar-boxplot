%&latex
%region
\documentclass[12pt]{article}
\usepackage{amsmath}
\usepackage{graphicx,psfrag,epsf}
\usepackage{enumerate}
\usepackage{natbib}
\usepackage{color,soul}
\usepackage{url} % not crucial - just used below for the URL 

%\pdfminorversion=4
% NOTE: To produce blinded version, replace "0" with "1" below.
\newcommand{\blind}{0}
\newcommand{\thetitle}{%
%endregion
%region Title
Radar-boxplot: a boxplot variation for multivariate classification data%
%endregion
%region
}

% DON'T change margins - should be 1 inch all around.
\addtolength{\oddsidemargin}{-.5in}%
\addtolength{\evensidemargin}{-.5in}%
\addtolength{\textwidth}{1in}%
\addtolength{\textheight}{1.3in}%
\addtolength{\topmargin}{-.8in}%


\begin{document}

%\bibliographystyle{natbib}

\def\spacingset#1{\renewcommand{\baselinestretch}%
{#1}\small\normalsize} \spacingset{1}


%%%%%%%%%%%%%%%%%%%%%%%%%%%%%%%%%%%%%%%%%%%%%%%%%%%%%%%%%%%%%%%%%%%%%%%%%%%%%%

\if0\blind
{
  \title{\bf \thetitle}
%endregion
%region Authors
\author{Author 1\thanks{
    The authors gratefully acknowledge \textit{please remember to list all relevant funding sources in the unblinded version}}\hspace{.2cm}\\
    Department of YYY, University of XXX\\
    and \\
    Author 2 \\
    Department of ZZZ, University of WWW}
  \maketitle
} \fi
%endregion
%region
\if1\blind
{
  \bigskip
  \bigskip
  \bigskip
  \begin{center}
    {\LARGE\bf Title}
\end{center}
  \medskip
} \fi

\bigskip
%endregion
\begin{abstract}
  The text of your abstract.  200 or fewer words.
\end{abstract}
%region Keywords
\noindent%
{\it Keywords:}  3 to 6 keywords, that do not appear in the title
\vfill

\newpage
\spacingset{1.45} % DON'T change the spacing!
%endregion

\section{Introduction}
\label{sec:intro}
GAP: As we inscrease our power to collect and analyze data, the datasets become larger, more complex and harder to understand. Visualizing classification multivariate data has been a non trivial task, relying on methods to reduce data dimensionality which may over simplify the problem and fail to represent the whole complexity of the dataset. 

In the realm of classification problem, there has been great advances, mainly due to increase in computational power, allowing to use complex machine learning algorithms. These algorithms have provided many advanced for interpreting and classifying data which were meaningless or could only be understood by the human eye. But those algorithms don't solve the problem of understanding how and why the data is meaningful, leaving selection of variables and understanding the problem a background issue. 

Exemplos de como os dados são analisados hoje em dia.


\section{Methods}
\label{sec:meth}
The radar-boxplot is plotted using the radar plot design, along with the traditional boxplot concepts proposed by \hl{FISHER: procurar}. 

While the radar plot can display multivariate data, it fails to represent groups, instead it plots individual features, whereas boxplot represent the distribution of a group but for a single variable. Merging the two concepts, we can visualize the distribution of a group for multiple variables in a single plot. 

The concept is simple. We create a polar plot, dividing it by the number of variables we need to represent, each axis is a variable. Then, it is possible to represent the Q25-Q75 as a single radial polygon, with a varying width along each axis, whereas the Q0-Q25 and Q75-Q100 will be represented by two other radial polygons along the margins of the first Q25-Q75 polygon. The outliers as defined in the boxplot and the median can also be represented as dots and a circular line.

The polar plot is drawn as ranging from 0 to 1, from the center out. The variables must be standardized to range from 0.1 to 1 to be drawn in the same scale. It is important that the minimum value is not 0, otherwise it will be difficult to visualize low values as they will overlap each other.

\section{Results}
\label{sec:results}
Gráficos
\begin{enumerate}
  \item Geral, radar-boxplot com multiplas classes: mostrando como o grafico consegue mostrar o panorama geral e revelar padroes que se confirmam pela acuracia atingida com os classificadores
  \item Matriz de erros para fazer o paralelo com as classes: Fazendo o paralelo com o grafico geral, demonstrando a correspondencia entre os padroes revelados pelo grafico e a confusao encontrada pelo classificador
  \item Gráfico radar-boxplot com feicoes classificadas erradas: demonstrando a possibilidade de verificar caso a caso se o individuo segue ou nao o padrao, apontando possibilidades para melhorar a classificacao
\end{enumerate}


\section{Discussion}
\label{sec:conc}
\bigskip
The radar-boxplot provides an intuitive way for visualizing multivariate classification data. 

\bibliographystyle{agsm}

\bibliography{Bibliography-MM-MC}
\end{document}